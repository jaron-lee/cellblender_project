\documentclass[a4paper]{article}
\title{Simulating calcium buffering and handling at nerve terminals with CellBlender/MCell}
\author{Jaron Lee}
\usepackage{mathtools}
\usepackage{graphicx}
\usepackage{siunitx}
\usepackage{natbib}
\usepackage{hyperref}
\usepackage{tabularx}
\usepackage{enumerate}
\usepackage{extarrows}
\usepackage[hmargin=3cm,vmargin=3.5cm]{geometry}
\usepackage{amssymb}
\usepackage{amsmath}
\usepackage{float}
\setlength\parindent{0pt} % sets indent to zero
\setlength{\parskip}{10pt} % changes vertical space between paragraphs
\renewcommand\thesubsection{\thesection.\roman{subsection}}
\makeatletter
\renewcommand*\env@matrix[1][*\c@MaxMatrixCols c]{%
  \hskip -\arraycolsep
  \let\@ifnextchar\new@ifnextchar
  \array{#1}}
\makeatother

\begin{document}
\maketitle

\section{Introduction}
This paper aims to investigate the calcium buffering aspect of synaptic transmission through a computational perspective. Utilising CellBlender with the MCell backend, we will create the model \textit{in silico} and simulate the activation and release of synaptic vesicles. As an example of the utility of these technologies, we present a bio-realistic model of a neuron synapse and demonstrate these results in animated form.

Blender is a 3D graphics modelling software which is free and traditionally used for applications such as digital media, games, visual art. MCell is a cell simulation command line utility which is capable of providing Monte Carlo simulations of a cell at the molecular level. CellBlender is a Blender addon which allows the user to access features within MCell through a Blender interface. Together with the general purpose language Python, this paper utilises these technologies to produce an accessible guide for simulating the calcium buffering and handling at nerve terminals.

The advantages of such an approach are many-fold. Firstly, the user does not need to be familiar with programming through a command line interface to do cellular level modelling, enabling a wider audience to access the power of a simulation environment. Secondly, the simulation results are presented in an animation, which is an information-dense medium and allows the user to quickly interpret the model results.

Finally, once the simulation has been set up, it is an easy task for the user to adapt the model parameters and geometric configurations to answer their scientific questions. 

All documentation, simulation files, scripts, tutorial instructions and this report can be found at \url{https://github.com/kanilor/cellblender_project}.

\section{Background Information}

\subsection{Synaptic Transmission - An Overview}
\begin{itemize}
    \item An action potential travels along the membrane of the presynaptic cell until it reaches the synapse.
    \item This depolarisation causes voltage gated calcium channels to open.
    \item Calcium ions flow from the extracellular space into the presynaptic terminal due to the concentration gradient.
    \item The high calcium concentration activates proteins on neurotransmitter vesicles, causing them to fuse with the synaptic cell membrane and emptying their contents (about 4000 molecules) into the synaptic cleft.
    \item The neurotransmitter molecules (AMPA, NMDA) bind to their respective receptors on the postsynaptic terminal. At least two neurotransmitter molecules  must bind to the receptor in order for it to activate. Most neurotransmitters are lost to the extracellular environment where they are retrieved by glial cells.
    \item Activation of a receptor will cause something to occur in the cell. For example, ligand-gated ion channels such as AMPA and NMDA receptors open ion channels which permit sodium, potassium and calcium ions to enter the postsynaptic terminal.
    \item The influx of ions will cause a change in voltage at the postsynaptic terminal, which produces a postsynaptic potential.

\end{itemize}

In our model we describe only the chemical aspects. The electrical aspects (i.e. the action potentials) are not described.

\subsection{Synaptic Structure}
The synapse consists of the presynaptic terminal (the bouton) and the postsynaptic terminal (the spin). 

Axons of neighbouring neurons are interspersed with small swellings along their length, and at the terminal end of the axon. These swellings are termed boutons or axonal varicosities, and at these locations synaptic connections may form. The bouton contains the synaptic vesicles and associated apparatus for vesicle release.

Dendrites of neurons are covered in spines, which are small protrusions consisting of a bulbous head and a thin body, as per \cite{Arellano:FrontNeurosci:2007}. Their primary role is to form excitatory synaptic connections, and as such the spine head region is covered in neurotransmitter receptors.


\subsection{Vesicles}

At each synapse, there are vesicles filled with neurotransmitters. Each vesicle is filled with about 4700 neutrotransmitter molecules, as per \cite{Bruns:Nature:1995}. During synaptic transmission, action potentials induce calcium influx into the bouton and induce vesicle fusion, which causes the vesicle to release its contents across the synaptic cleft.

The majority of vesicles are undocked (not at the presynaptic membrane) and docking is a process which takes several minutes. An estimate by CITATION gives about 250 vesicles in the active docked phase, and 500 vesicles that are undocked.

Docked vesicles can fuse with the snyaptic membrane. SNARE (soluble N-ethylaleimide-sensitive factor attachment receptor) proteins hold the vesicle in place. The SNARE complex consists of the subunits synaptobrevin, syntaxin and SNAP-25. The SNARE complex acts to 'unzip' the vesicle when an appropriate protein is activated. For our synapse, this protein is synaptotagmin-1, which is receptive to calcium ions. 

Each synaptotagmin-1 unit has four smaller subunits. Each subunit may be activated by a single calcium ion. The consensus on the number of calcium ions required to activate the synaptotagmin-1 unit is 2, according to \cite{Dittrich:BiophysJ:2013}.

There is no widely accepted consensus on how many synaptotagmin-1 proteins are required to activate the SNARE complex and cause vesicle fusion. For example, it is cited in \cite{vandenBogaart:NatStructMolBiol:2010} that a single SNARE complex is sufficient for vesicle release. However, \cite{Dittrich:BiophysJ:2013} state that at least three SNARE complexes are required for vesicle release.

Based on the existing literature, we suggest the following compromise. At least two calcium ions must bind to the synaptotagmin protein in order for it to engage with and activate the SNARE complex. A vesicle will require three such activated SNARE complexes in order for vesicle release (or fusion) to occur.

\subsection{Receptors}
We are interested in investigating the glutamate neurotransmitter, and the corresponding receptors are the AMPA and NDMA receptors.

The AMPA receptor is a heterotetramer, and is composed of four types subunits - GluR1, GluR2, GluR3 and GluR4. It is a fast-opening receptor, with a response time of CITATION.

\section{The Modelling Approach}

The specific details on how to use CellBlender to create a model achieving the goals outlined above are contained in a separate document "A Guide on Modelling Synapses with CellBlender and MCell", which the reader should refer to for all technical details.

Here, we will discuss the scientific details of the modelling approach.

\subsection{Summary of the Modelling Approach}
The modelling occurs in two phases - presynaptic, and postsynaptic.

In the presynaptic phase, we simulate calcium ions moving through the presynaptic bouton and then binding to the SNARE complexes (in the model, we treat the SNARE-synaptotagmin interaction as a single unit). After each SNARE complex is activated, the model produces a tag molecule indicating the event has occurred. The tag molecule has no model properties otherwise.

A script (inspired by the approach taken in \cite{ma2014quantitative}) was written to scan over the model files and detect the times at which the tag molecules are produced. After a certain number of tag molecules accumulate on a vesicle, it is then declared to be activated, and this time is reported in the script. The user can then manually enter these times into the second phase of the simulation.

In the postsynaptic phase of the simulation, we simulate the vesicle fusion using the vesicle fusion times produced by the script analysis. The neurotransmitter molecules in the vesicle are created in the model at the times that the vesicles are released, and they can diffuse through the synaptic cleft to the receptors on the spine head.

Finally, these two phases are combined to provide a single contiguous simulation of calcium handling at the synaptic terminal.

The instructions are supplied as a PDF document, as well as an IPython Notebook. The modelling approach draws heavily from the guide of \cite{Czech:MethodsMolBiol:2009} in the initial setup. However, the guide was written for an earlier iteration of CellBlender called DREAMM. The tutorial document we have provided updates and improves upon that guide for the current CellBlender platform.

\subsection{Scale, Size and Time}
In order for the simulation to be realistic, the dimensions and units within Blender must match those of the MCell simulation. It is possible to create a shape of any size in Blender, but this size may not be appropriate for the simulation scale.

In MCell, the units of space are $\SI{}{\micro\metre}$, the unit of time is $\SI{}{\second}$, diffusion constants are $\SI{}{\centi\metre\squared\per\second}$. Unimolecular reactions have rate units $\SI{}{\per\second}$, volume-volume or volume-surface bimolecular reactions have units $\SI{}{\per\mole\litre\per\second}$ and surface-surface bimolecular reactions have units $\SI{}{\micro\metre\squared\per\mole\litre\second}$. One Blender unit is equivalent to $\SI{1}{\micro\metre}$ in MCell. 

It is also important to note that CellBlender does discrete simulation - the state of the modelling environment is calculated at a certain time. A short time interval is allowed to pass, and CellBlender again recalculates the state of the modelling environment. The end product is a series of frames which can be played sequentially as an animation.

\subsection{Translating the Biology into Blender}
All models are simplifications of reality, and this model is no different. In this section we outline the simplifications made for the purposes of modelling.

We simulate only a single synaptic connection. This consists of a single presynaptic bouton, and a single postsynaptic receptor. These are created \textit{in silico} using Blender geometry objects, and are static objects within the simulation. The surfaces of these objects may be defined to prevent molecules passing through.

Vesicles are simulated as spherical Blender objects situated within the presynaptic bouton. These are also static objects within the simulation. Vesicle fusion is defined in a stylised manner as follows. On the vesicle surface we define several SNARE molecules. Each SNARE molecule needs to accept 2 calcium ions to become 'activated'. In order for vesicle fusion to occur, at least 3 molecules must be activated. Note that the actual process of vesicle fusion is not simulated - this functionality is not directly implemented in CellBlender, and an external script was written to simulate this.

Molecules in CellBlender are point objects with a defined interaction radius (left at defaults). No steric interactions were simulated. Molecules react in user-defined reactions, and do not have any intrinsic chemical, biological or steric properties. 

\section{Making the Model Realistic}
In this section we have sought to specify the quantities and parameters used in our model to the extent that these data exist. It is important that our model mimic the reality to a reasonable amount to ensure the integrity and usefulness of the model.

\subsection{Model Components}
The model as implemented in the accompanying guide consists of the following components:
\begin{enumerate} 
    \item Presynaptic bouton
    \item Spinehead
    \item Voltage-gated calcium channels (2)
    \item Vesicles (2)
    \item Glial cells
\end{enumerate}

It consists of the following molecules/proteins:
\begin{enumerate}
    \item VGCC - the calcium channels responsible for permitting flow of calcium into bouton. Has open and closed states.
    \item Ca - calcium ion
    \item CaBS - a SNARE complex which has a single calcium ion bound
    \item TAG - a SNARE complex which has two calcium ions bound
    \item NT - a neurotransmitter molecule (represents glutamate)
    \item LGIC - a neurotransmitter receptor residing on spine head (receptive to NT). Has open and closed states.
\end{enumerate}

\subsection{Model Equations}
The model as implemented in the accompanying guide consists of the following reactions:
\begin{table}[H]
\begin{tabular}{ll}
Equation & Description \\ \hline
VGCC\_C $\to$ VGCC\_O & Calcium channel opening \\
VGCC\_O $\to$ VGCC\_C & Calcium channel closing \\
VGCC\_O $\to$ VGCC\_O + Ca & Calcium influx into bouton \\
Ca + CaBS $\to$ CaBS\_Ca & First calcium binding to synaptotagmin/SNARE complex  \\ 
CaBS\_Ca + Ca $\to$ TAG & Second calcium binding to synaptotagmin/SNARE complex  \\
NT + LGIC\_C $\to$ LGIC\_O& Neurotransmitter binding to receptor \\
\end{tabular} 
\end{table}


\subsection{Reaction Specifications}
We can summarise the model mechanism with chemical reactions. Some of these reactions have been highly simplified for simulation ease.

Recall that our vesicle fusion mechanism has been simplified to a reaction involving SNARE complexes and calcium ions. We can take the rate of calcium binding to synaptotagmin as a proxy for this value. The authors in \cite{ma2014quantitative} provide a rate value of $\SI{1e8}{\per\mol\litre\per\second}$ for what they call the 'synaptotagmin-like calcium binding site'. 

The rate of calcium influx into the bouton was difficult to determine, due to the simplified mechanism of the calcium channel in the model. We revert to the reference guide by \cite{Czech:MethodsMolBiol:2009} and use their rate value of $\SI{1e3}{\per\mole\litre\per\second}$.

The vesicle unzip time has been estimated at $\SI{200}{\micro\second}$, from \cite{Llinas:TheSquidGiantSynapseAModel:1999}. This value is a rough approximation only, since the homeostasis conditions of a squid differ greatly from that of a human. We assume that the unzipping occurs linearly over this time interval.

The rate of diffusion of the neurotransmitter glutamate was estimated at $\SI{4e-6}{\centi\metre\squared\per\second}$. The rate of glutamate receptor binding was estimated at $\SI{4.6e5}{\per\mol\litre\per\second}$. These values were taken from \cite{rusakov2001role} by averaging the upper and lower bounds on the diffusion and rate constants.

The rate of diffusion of calcium was estimated at $\SI{5.3e-6}{\centi\metre\squared\per\second}$ and was drawn from \cite{Dittrich:BiophysJ:2013}.
\begin{table}[H]
\begin{tabular}{lll}
Parameter & Value & Reference \\ \hline
Rate of calcium influx   &  $\SI{1e3}{\per\mole\litre\per\second}$      &    \cite{Czech:MethodsMolBiol:2009} \\
SNARE complex binding rate & $\SI{1e8}{\per\mol\litre\per\second}$ & \cite{ma2014quantitative}\\
Glutamate binding rate & $\SI{4.6e6}{\per\mol\litre\per\second}$ & \cite{rusakov2001role} \\
Rate of glutamate diffusion & $\SI{4e-6}{\centi\metre\squared\per\second} $      &\cite{rusakov2001role} \\
Rate of calcium diffusion & $\SI{5.3e-6}{\centi\metre\squared\per\second} $      &\cite{Dittrich:BiophysJ:2013} \\
Vesicle unzip time & \SI{200}{\micro\second} & \cite{Llinas:TheSquidGiantSynapseAModel:1999}\\
\end{tabular}
\end{table}

\subsection{Object Specifications}
Morphology data on the size of the bouton and the sizes and quantities of the synaptic vesicles was kindly provided by CITATION. 

The mean estimate for the bouton volume was given as $\SI{0.36}{\micro\meter\cubed}$. We can approximate the radius of the bouton by treating it as a hemisphere to obtain a value of $\SI{0.7}{\micro\metre}$. 

The mean estimate for the synaptic vesicle radius was given as $\SI{0.017}{\micro\meter}$.

The size of the postsynaptic spine head was estimated to have the same dimensions as the presynaptic bouton.

Finally, we estimated the cleft width at a constant $\SI{0.023}{\micro\meter}$ - this is an average of lateral and central cleft width measurements.

\begin{table}[H]
\begin{tabular}{lll}
Parameter & Value & Reference \\ \hline
Estimate for bouton volume&  $\SI{0.36}{\micro\meter\cubed}$& CITATION \\ 
Derived estimate for bouton radius & \SI{0.7}{\micro\meter} & Derived using hemisphere approximation\\
Estimate for synaptic vesicle radius & $\SI{0.017}{\micro\meter}$ & CITATION \\ 
Synaptic cleft width &$\SI{0.023}{\micro\meter}$& CITATION\\
\end{tabular}
\end{table}

\subsubsection{Molecule Quantity Specifications}
From \cite{Bruns:Nature:1995} we have drawn an estimate of 4700 neurotransmitter molecules per vesicle (in this case the data is drawn from a vesicle containing the glutamate molecule).

From \cite{Stricker:JPhysiol:1996} we have taken the estimate of 100 neurotransmitter receptors. The data was originally taken from a mouse hippocampus. 

CITATION estimated a mean of $750$ vesicles at the bouton; however we have only modelled two due to computing constraints.

From \cite{Wilhelm:Science:2014} we have drawn an estimate of 26000 SNAP-25 proteins in a synapse. There are two SNAP-25 proteins per SNARE complex, and about 750 vesicles in total. From this we can estimate about 15 SNARE complexes per vesicle. 

\cite{Dittrich:BiophysJ:2013} state that the number of calcium ions required to activate a synaptotagmin protein (and hence activate the SNARE complex) is a minimum of 2. They also state that 3 such activated SNARE complexes are required to cause vesicle release. 

\begin{table}[H]
\begin{tabular}{lll}
Parameter & Value & Reference \\ \hline
Number of Neurotransmitter molecules per vesicle & 4700&\cite{Bruns:Nature:1995} \\
Number of Neurotransimtter receptors per spine head & 100 & \cite{Stricker:JPhysiol:1996} \\
Number of SNARE complexes per vesicle & 15 & \cite{Wilhelm:Science:2014} \\ 
Number of calcium ions to activate synaptotagmin/SNARE & 2 & \cite{Dittrich:BiophysJ:2013} \\
Number of SNAREs to induce vesicle fusion & 3 & \cite{Dittrich:BiophysJ:2013} \\  
Number of vesicles & 750 & CITATION\\
\end{tabular}
\end{table}

\section{Results}
A model of a synaptic connection was implemented according to the documentation in the accompanying reference "A Guide on Modelling Synapses with CellBlender and MCell". This model was created with the intention of making it a useful approximation of reality. To that end, we have drawn model parameters and dimensions from the neuroscience literature, and these are documented above.

The model consists of a single synaptic connection between a presynaptic axonal bouton and a postsynaptic spine head. The bouton contains two vesicles, both of which are considered active or docked (i.e. ready for release). We run the simulation till the first vesicle is releas.

When the simulation is running, the calcium ions are randomly generated and allowed to permeate through the model. In this particular instance of the simulation, the vesicle activation time was $\SI{461}{\micro\second}$. 


\section{Conclusion}

\section{Acknowledgements}
The author would like to thank Jacob Czech of the Pittsburgh Supercomputing Centre for his direct assistance in resolving some technical questions regarding the CellBlender addon, as well as Markus Dittrich of the same institution for directing the authors queries to Mr. Czech. The author would also like to thank CITATION for allowing the use of their pre-publication morphology data on L5 synaptic connections.


\bibliography{report_doc}{}
\bibliographystyle{cell}



\end{document}

